\chapter{The Non Preemptable Section Latency}

\definition{Non-Preemptable Section Latency}{Delay between time when an event is generated and when the kernel handles it.
\begin{itemize}
    \item Due to non-preemptable sections in the kernel, which delay the response to hardware interrupts
    \item Composed by various parts: interrupt disabling, bottom halves delaying,\dots
    \item Depends on how the kernel handles the various events
\end{itemize}}

The non-preemptable section latency $L^{np}$ is given by the sum of different components:
\begin{enumerate}
    \item Interrupt disabling
    \item Delayed interrupt service
    \item Delayed scheduler invocation
\end{enumerate}
The first two are mechanisms used by the kernel to guarantee the consistency of internal structures.\\
The third mechanism is sometimes used to reduce the number of preemptions and increase the system throughput.

\section{Interrupt disabling}
Before checking ifg an interrupt is fired, the CPU checks if interrupts are enabled. Every CPU has some protected intructions (\texttt{STI/CLI} on x86) for enabling/disabling interrupts.\\
In modern system, only the kernel (or code running in KS) can enable/disable interrupts.\\
Interrupts disabled for a time $T^{cli} \rightarrow L^{np}\ge T^{cli}$\\
Interrupt disabling is used to enforce mutual exclusion between sections of the kernel and ISRs.
\section{Dealyed Interrupt Service}
When the interrupt fires, the ISR is ran, but the kernel can delay the interrupt service some more\dots
\begin{itemize}
    \item ISRs are generally small, and do only few things
    \item An ISR can set some kind of software flag, to notify that the interrupt fired
    \item Later, the kernel can check such flag and run a larger (and more complex) interrupt handler
\end{itemize}
Hard IRQ handlers (ISRs) vs "Soft IRQ handlers".

The advantages of "soft IRQ handlers" are:
\begin{itemize}
    \item ISRs generally run with interrupts disabled
    \item Soft IRQ handlers can re-enable hardware interrupts
    \item Enabling/Disabling soft handlers is simpler/cheaper
\end{itemize}
Disadvantages:
\begin{itemize}
    \item Increase NP latency: $L^{np} >> T^{cli}$
    \item Soft IRQ handlers are often non-preemptable increasing the latency for other tasks too\dots
\end{itemize}
\section{Delayed scheduler invocation}
Scheduler invoked when returing from KS to US.
Sometimes, return to US after a lot of activities
\begin{itemize}
    \item Try to reduce the numbert of KS - US switches
    \item Reduce the number of context switches
    \item Throughput vs low latency
\end{itemize}
ISR executed at the correct time, soft IRQ handler ran immediately, but scheduler invoked too late

\section{Summary}
$L^{np}$ depends on some different factors.\\
In general, no hardware reasons, it almost entirely depends on the kernel structure: non-preemptable section latency is generally the result of the strategy used by the kernel for ensuring mutual exclusion on its internal data structures.

To analyze/reduce $L^{np}$, we need to understand such strategies.\\
Different kernels, based on different structures, work in different ways.\\
Some activities causing $L^{np}$ are: interrupt handling (device drivers) and management of the parallelism