\chapter{Real-Time scheduling and analysis}

\section{Real-Time Scheduling}
\subsection{Definitions}
Definitions:
\begin{itemize}
\item \side{Algorithm}: logical procedure used to solve a problem
\item \side{Program}: formal description of an algorithm, using a \side{programming language}
\item \side{Process}: instance of a program (program in execution)
\begin{itemize}
\item Program: static entity
\item Process: dynamic entity
\end{itemize}
\item The term \side{task} is used to indicate a schedulable entity (either a process or a thread)
\begin{itemize}
\item \side{Thread}: flow of execution
\item Process: flow of execution + private resources (address space, file table, etc...)
\end{itemize}
\end{itemize}

\subsection{Executing Concurrent Tasks}
Tasks do not run on bare HW...
\begin{itemize}
\item How can multiple tasks execute on one single CPU?
\item The \side{OS kernel} creates the illusion of having more CPUs, so that multiple tasks execute in parallel
\begin{itemize}
\item Tasks have the illusion of executing concurrently
\item a dedicated CPU per task
\end{itemize}
\end{itemize}

\subsection{Scheduling Concurrent Tasks}

\subsection{Scheduler}

\subsection{Cyclic Executive Scheduling}
\subsection{Fixed Priority Scheduling}
\subsection{Priority Assignment}
\subsection{Optimal Priority Assignment}
\subsection{Analysis}

\section{Real-Time Scheduling analysis}